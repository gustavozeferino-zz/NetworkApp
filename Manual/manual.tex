\documentclass[10pt]{apa}

% Load packages:
\usepackage[english]{babel}
\usepackage[a4paper]{geometry}
\usepackage{amsmath}
\usepackage{amssymb}
\usepackage{graphicx}
\usepackage{listings}
\usepackage{apacite}
\usepackage{color}
\usepackage{url}

\begin{document}

\title{NetworkApp: User Manual}
\author{Jolanda J. Kossakowski}
\affiliation{University of Amsterdam}
\date{\today}
\maketitle

\tableofcontents

\clearpage

\section{Introduction}

Network analysis has gained popularity in recent years. By the development of R-packages like \emph{qgraph}\cite{epskamp2012} and \emph{igraph} \cite{csardi2006}, it has become easier to visualize network structures and apply different estimation methods. The drawback of this is that users have to learn \texttt{R} in order to use these packages. The goal of this application was to design and programme a web application that enables users to visualize and analyze networks without having to know the R-language itself. This manual will aid users and explain all the functions and options that are currently available within the application. 

\section{Uploading Data}

In the grey left panel of the application, users can upload either a `.txt' or a `.csv' file. Users only have to specify the type of data that they are uploading, how missing values, if present, are coded within the data file, and whether character variables (variables that contain words instead of numbers) should be coded as \emph{factors}. In the case that the user discriminated between different scores by using words, this option should be checked.

\includegraphics[width=0.3\textwidth]{StringAsFactors}


\subsection{Data Types}

The user can enter three different types of data. The first option is `raw data'. 

\section{Network Esthetics}

\section{Centrality Analyses}

\section{Clustering Analyses}

\section{Network Comparison}

\newpage
\bibliographystyle{apacite}
\bibliography{bibfile.bib}


\end{document}